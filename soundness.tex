\section{Implementing Logical Machinery \& Soundness}
We build our program logic, as an instantiation of Iris~\cite{iris}, and build our modal abstractions on top of it. 
\mytodo{we should put some stuff here}
\subsection{Soundness}
\label{sec:soundness}
Our logic, operates on the machine state, which means we do not need to augment the machine state. The invarian that we pick, \textit{central invariant} ($\mathcal{I}$), is just semantic interpretation of stores in the machine state, $\sigma$.R and $\sigma$.M. This semantic interpratation ensures the correct lifting of mappings in the machine state to the assertions that the client of our logic uses, i.e. points-to assertions that are defined as the ownership of a fragment of the logical state.

We prefer to skip explaining the steps used in instantiation of Iris because it is an almost standard procedure, has already been explained for many other logic \ref{}, and we are concerned with the page-count limitation. However, it is worh noting that once you instantiate Iris for your language, it comes with the semantic definition for the weakest-precondition which we can refactor into Hoare triples, $\triple\pre\instr\post$, to specify our  selected \textsf{AMD64} instructions shown in Figure \ref{fig:wpdamd}, and show that these triples are sound. 
\subsection{The Soundness Statement}
\label{def:soundness:statement}
The operational semantics of our simple lang just executes sequences of instructions in our x86-64 model. Therefore, our soundness argument is to show that any execution composed of instructions in our machine model (some of which are shown in Figure \sref{sec:semantics}) does not end-up in a invalid state.
\begin{figure*}
\small
\begin{mathpar}
\inferrule[Skip]{}{
  \triple\post\iskip\post
}

\inferrule[Seq]{
  \triple\pre{\instr_1}\midpoint \\
  \triple\midpoint{\instr_2}\post
}{
  \triple\pre{(\iseq{\instr_1}{\instr_2})}\post
}
\end{mathpar}
\caption{Structural Rules for Executing Instructions}
\Description{Reasoning Rules}
\label{fig:structural}
\end{figure*}

\begin{theorem}[Soundness of the Logic]
  \label{th:adequacy}
 Together with the assumptions on the initial state hold,
 the execution of the instruction~$\instrs$, beginning with this initial state, cannot result in a configuration where the execution is stuck.
\end{theorem}
which states that if the program~$\instr$, with the given \textsf{valid\_init} asserting a valid state initialization, satisfies a semantic Hoare
triple, then this program cannot crash: by a direct consequence of Iris's adequacy theorem~\cite[\S6.4]{iris}.

Moreover we need to show the validity of each rules in Figures \fref{fig:reasoning} and \fref{fig:laws}.
\begin{theorem}[Validity of the Reasoning Rules]
\label{th:validity}
  Each of the rules in Figures~\ref{fig:wpdamd}
  and~\ref{fig:structural} is valid.
\end{theorem}
Due to the space limits in this paper, we do not mention each proof for the rules in Figures \ref{fig:wpdamd} and \ref{fig:structural} within this section, but we provide mechanized proofs for all these in Coq as a part of our artifact submission.
However, we would like to give the definitions and constructions used in our proofs, and would like to give an outline of paper proof for \TirNameStyle{WriteToRegFromVirtMem} in Figure \ref{fig:wpdamd} within this section.

Together, Theorems~\ref{th:adequacy} and~\ref{th:validity} guarantee that, if
the Hoare triple $\textsf{valid\_init }\instrs\;\iTrue$ can be obtained by applying
the reasoning rules of our logic, then the program~$\instrs$ is safe.

\subsection{Logical Constructions}
\label{sec:invariant}

We already have the physical and logical stores and a simple invariant between them. Now, we can rely on the following Assumption \ref{assumption} from Iris to utilize its logical constructions.
% The predicate gen_heap_interp.
\newcommand{\genheapinterp}[1]{\mathit{Heap}\;#1}
\newcommand{\genmemheapinterp}[1]{\mathit{MemHeap}\;#1}
% Our predicate pred (defined in ph.v), expanded.
\newcommand{\pred}[1]{\ownGhost\gammaPred{\authfull{(\mapone\predstore)}}}
% A notation for assigning fraction 1 to every element of \predstore.
\newcommand{\mapone}[1]{1.#1}
% The predicate mapsfrom_exact, expanded.
\newcommand{\mapsfromexact}[3]{
  \ownGhost\gammaPred{\authfrag{\singletonMap{#1}{(#2, #3)}}}
}
% A metavariable for a share.
\newcommand{\sh}{L'}
% The predicate mapsfrom, expanded.
\newcommand{\mapsfromdef}[3]{
  \exists\sh.\;
  \mapsfromexact{#1}{#2}{\sh} \star \pure{\sh \subseteq #3}
}

\begin{assumption}
\label{assumption}
Iris defines two pieces of ghost state
\begin{enumerate}
\item  defines a predicate $\genheapinterp\store.\mathcal{R}$
  that ties a store~$\store.\mathcal{R}$ to this ghost state,
  and defines the points-to assertion $\ppointsto\rg\rv\qfrac\rpts$
  in terms of this ghost state.
  This is visible in the paper~\cite[\S6.3.2]{iris}
  and in Iris's \texttt{gen\_heap} library~\cite{genheap}.
  %
  We re-use this machinery without change,
  so we do not repeat these definitions.
  We mention the predicate $\genheapinterp\!$
  in our own invariant (Definition~\ref{def:invariant}),
  where it is applied to the \logical store~$\store$.
\item unlike the points-to relation obtained by direct interpretion of $\genheapinterp\store.\mathcal{R}$ using Iris, the existing \textsf{gen\_heap\_interp},
  does not directly helps introducing the ghost we define a new algebra (\textsf{gen\_mem\_UR}) for abstracting the nested maps due to different levels of masking in memory mappings. 
\end{enumerate}
\end{assumption}

\begin{definition}[Ghost State - Memory with Nested Mappings]
We define our custom-tailored algebra 
\newcommand\fpfn{\rightarrow_{\textrm{fin}}}
\( \textsf{gen\_memUR} \stackrel{def}{=}
  \authm(\;
  \Locft \;\fpfn\;
  (\Loctw \;\fpfn\;  (\textsc{Frac }, \mathord{+}) \times (\textsc{Agree } \Loc,\mathord{=}) )
  \)
  for our nested memory mapping abstracting two different machine word masking. Then, we register our \emph{authoritative camera} \cite[\S6.3.3]{iris} from this algebra
\[\textsf{inG } \Sigma \textsf{ gen\_memUR} \]
As a final step, we allocate an empty ghost cell $\gamma$ at the beggining of a program execution to store the element of the monoid $\textsf{inG } \Sigma \textsf{ gen\_memUR}$.
\end{definition}

\begin{definition}[Ghost State - Register Mappings]
We allocate $\theta$ ghost cell which stores an
element of the monoid 
\newcommand\fpfn{\rightarrow_{\textrm{fin}}}
\(
  \authm(\;
    \regset \;\fpfn\;
    (\textsc{Frac}, \mathord{+})
    \times
    (\regvaltype, \mathord{=})
  \;)
\)
% \emph{authoritative camera}
\cite[\S6.3.3]{iris}.
\end{definition}

\begin{definition}[Central Invariant]
\label{def:invariant}
The central invariant of our logic is, due to lack of need for augmenting the machine state, simply the state interpretation \cite[\S7.3]{iris} : 
\[
\centralinvariant\;\store \triangleq
\left\{
\def\arraystretch{1.2}
\begin{array}{l@{\quad\star\quad}l@{\quad}l}
  \genheapinterp\store &  \\
   \genmemheapinterp\store
\end{array}
\right.
\]
\end{definition}

As noted earlier (Assumption~\ref{assumption}), the predicate
$\genheapinterp\!$ and the points-to assertion for register mappings is inherited from Iris. What
this means, as far as we are concerned, is that the points-to assertion is
defined in a standard manner: $\genheapinterp\!$ is applied to the physical store~$\store$, as is usually the case~\cite[\S6.3.2]{iris}.
Likewise, we define nested interpretation $\genmemheapinterp\!$ in terms of $\genheapinterp\!$ to obtain physical memory points-to relation.

In fact, long story short, all assertions in our logic, i.e. points-to assertions for registers and memory, really about the logical store, which is in our case identical to the physical store, so that we can reason about the physical store.

