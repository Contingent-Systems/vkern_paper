\section{Implementing Logical Machinery \& Soundness}
We instantiate our program logic, $[@]$, as an instantiation of Iris~\cite{iris}, and build our modal abstractions on top of it. 
\subsection{BI-Lifting for Address Space Modality}
\label{sec:bilifting}
\mytodo{we should put some stuff here}
\subsection{Soundness}
\label{sec:soundness}
Our logic, $[@]$, operates on the machine state, which means we do not need to augment the machine state, i.e. our logical state is exact logical representation of our machine state. Consequently, we do not require an invariant, $\mathcal{I}$, except from the fact that logical/physical associated maps representing the register/memory mappings have the same domain. This ensures the correct lifting of mappings in the machine state to the \emph{assertions} that the client of our logic uses, i.e. points-to assertions that are defined as the ownership of a fragment of the logical state.

The crucial pragmatic aspect of instantiating Iris for your reasoning principles \todo{Tired stop here.} 
Once these steps have been carried out, Iris provides a semantic definition of
the Hoare triple $\triple\pre\instr\post$. We can then use Hoare triples to
state the reasoning rules of ...., to prove that
these rules are valid, and to state and prove the fact that ... is sound.
This is the topic of the next subsection (\sref{def:soundness:statement}).
%
After presenting these results, in \sref{sec:invariant}, we come back to the
three steps listed above and describe them at an informal level, without
diving into the concepts of Iris, which would be difficult to explain in a
thorough and concise way.
% For more information on Iris, the reader is referred to Jung \etal.'s
% paper~(\citeyear{iris}).

\subsection{The Soundness Statement}
\label{def:soundness:statement}

The operational semantics of ...lang guarantees that ... (\sref{sec:valid}). A~program that attempts to ... stuck~(\sref{sec:instrsemantics}).
% Thus, the space consumption of every \spacelang program is bounded.
Thus, in order to prove that ... is sound, all we have to do is prove that
it rules out the possibility of a runtime failure. This is guaranteed by the
following theorem:

\begin{theorem}[Soundness of the Logic ...]
\label{th:adequacy}
 The execution of the instruction~$\instrs$, beginning in an initial state, cannot result in a configuration where a thread is stuck.
\end{theorem}

This theorem states that if the program~$\instr$ satisfies a semantic Hoare
triple, then this program cannot crash. In
particular, it cannot reach a situation where DOING X ....
%
The proof of this theorem is short: it is a direct consequence of Iris's
adequacy theorem~\cite[\S6.4]{iris}.

Separately, we prove that the reasoning rules of~our logic are valid, that is,
each rule is a valid lemma:

\begin{theorem}[Validity of the Reasoning Rules]
\label{th:validity}
  Each of the rules in Figures~\ref{fig:laws}
  and~\ref{fig:reasoning} is valid.
\end{theorem}

The proof of this theorem represents more work: one lemma per reasoning rule
is required, not to mention a large number of auxiliary lemmas. The proofs of
these lemmas are fairly uninteresting; the key insights lie in the definitions
presented in the next subsection (\sref{sec:invariant}).

Together, Theorems~\ref{th:adequacy} and~\ref{th:validity} guarantee that, if
the Hoare triple $\iTrue\instrs\iTrue$ can be obtained by applying
the reasoning rules of our logic, then the program~$\instrs$ is safe, that is, it cannot crash ....

\subsection{Key Definitions and Invariants}
\label{sec:invariant}

For the definitions that follow, we need some more terminology about stores.

Next, we must define the relation that ties the physical store to the
\logical store, two concepts that we have informally described earlier
(\sref{sec:logical}). From here on, we write~$\store$ for the physical
store and~$\logicalstore$ for the \logical store.

Equipped with the above definitions, we are now prepared to enter the realm of
Iris and perform the three steps described at the beginning of
\sref{sec:soundness}, that is, define the ghost state, the central invariant,
and the assertions of our .. . We give the definitions first and explain them
afterwards.

\begin{comment}
% fp: this is the beginning of an explanation of ghost state in Iris
%     but it quickly grows too long, even though many aspects are missing.
A ghost memory cell in Iris is a memory cell that does not exist at runtime.
The ownership and the content of a physical memory cell are described via a
points-to predicate such as $\loc\pointsto\val$. Similarly, the ownership and
content of a ghost cell are described by an Iris assertion, usually written
$\ownGhost\gamma{a}$, where $\gamma$ is the name of the ghost cell and $a$ is
its content. Whereas the content of a physical memory cell is a
programming-language value~$\val$, the content of a ghost cell inhabits a
mathematical structure, a \emph{camera}~\cite[\S4.4]{iris}, which the user
chooses when the ghost cell is created. A camera is equipped with a
composition operation~$\cdot$ and the ghost state assertion satisfies the law
$\ownGhost\gamma{a\cdot b} \equiv \ownGhost\gamma{a} \star
\ownGhost\gamma{b}$. Thus, the choice of a suitable camera determines in what
ways the ghost state can be split.

A key Iris idiom involves the use of the \emph{authoritative
  camera}~\cite[\S6.3.3]{iris} together with an Iris invariant.
The elements of the authoritative camera include
\emph{authoritative elements} $\authfull{c}$ and
\emph{fragmentary elements} $\authfrag{b}$.
The composition law is defined in such a way that
there always exists at most one authoritative element
and the composition of all fragmentary elements in existence is
contained in the authoritative element, etc.
\end{comment}
% The predicate gen_heap_interp.
\newcommand{\genheapinterp}[1]{\mathit{Heap}\;#1}
% Our predicate pred (defined in ph.v), expanded.
\newcommand{\pred}[1]{\ownGhost\gammaPred{\authfull{(\mapone\predstore)}}}
% A notation for assigning fraction 1 to every element of \predstore.
\newcommand{\mapone}[1]{1.#1}
% The predicate mapsfrom_exact, expanded.
\newcommand{\mapsfromexact}[3]{
  \ownGhost\gammaPred{\authfrag{\singletonMap{#1}{(#2, #3)}}}
}
% A metavariable for a share.
\newcommand{\sh}{L'}
% The predicate mapsfrom, expanded.
\newcommand{\mapsfromdef}[3]{
  \exists\sh.\;
  \mapsfromexact{#1}{#2}{\sh} \star \pure{\sh \subseteq #3}
}


\begin{assumption}
\label{assumption}
  Iris defines a certain piece of ghost state,
  defines a predicate $\genheapinterp\store$
  that ties a store~$\store$ to this ghost state,
  and defines the points-to assertion $\loc\fpointsto\qv\blk$
  in terms of this ghost state.
  This is visible in the paper~\cite[\S6.3.2]{iris}
  and in Iris's \texttt{gen\_heap} library~\cite{genheap}.
  %
  We re-use this machinery without change,
  so we do not repeat these definitions.
  We mention the predicate $\genheapinterp\!$
  in our own invariant (Definition~\ref{def:invariant}),
  where it is applied to the \logical store~$\store$.
\end{assumption}

\begin{definition}[Ghost State]
...
stores an element of the monoid
\newcommand\fpfn{\rightarrow_{\textrm{fin}}}
\(
  \authm(\;
    \Loc \;\fpfn\;
    (\textsc{Frac}, \mathord{+})
    \times
    (\textsc{Multiset}(\Loc), \mathord\uplus)
  \;)
\)
% \emph{authoritative camera}
\cite[\S6.3.3]{iris}.
\end{definition}

\begin{definition}[Central Invariant]
\label{def:invariant}
The central invariant of our logic,
or \emph{state interpretation invariant} \cite[\S7.3]{iris},
is the following Iris assertion:
\[
\centralinvariant\;\store \triangleq
\left\{
\def\arraystretch{1.2}
\begin{array}{l@{\quad\star\quad}l@{\quad}l}
  \genheapinterp\store & \star \\
   \genheapinterp\store & \star
\end{array}
\right.
\]
\end{definition}

\begin{definition}[Assertions]
\label{def:assertions}
  The assertions of our ... are defined as follows:
  \begin{enumerate}
  \item
  % The definition of the points-to predicate is inherited from gen_heap.
  % It is implicitly parameterized by a ghost cell that holds the heap.
  The \emph{points-to assertion} ..
  is inherited from Iris (Assumption~\ref{assumption}).
  \item
  The \emph{map-of-map} for a single physical memory location,
  
  \item
  The \emph{location-virtualization-modal}
  \item
  The \emph{pointed-by assertion} .
  \end{enumerate}
\end{definition}

% What do these definitions mean?

Let us try and explain the most important aspects of these definitions.

In Iris, the state interpretation invariant~$\centralinvariant\;\store$ is an assertion that
holds of the \emph{physical store}~$\store$ in between every two steps of
computation. It is used by Iris in the definition of Hoare triples; we inherit
this definition from Iris.

Our definition of $\centralinvariant\;\store$ begins with an existential quantification over
a \emph{\logical store}~$\logicalstore$.
..........
%.......
