\section{Program Logic for Location Virtualization}
\label{sec:logic}
In this section, we describe our high-level reasoning principles for location virtualization whose detailed justification is shown in \sref{sec:soundness}. The principles shown here are mainly shaped around understanding and expressing the fact related the virtual address space, most specifically virtualization and translation of memory locations .
\subsection{Points-To Assertions}
\label{sec:pointsto}

We present three main points-to relations as part of utilization of separation logic \ref{} as an ambient logic:
\begin{enumerate}
\item Physical address points-to, $\pfpointsto\locsf\locsf\qfrac\ppts$
\item Register points-to, $\pfpointsto\rg\rv\qfrac\rpts$
\item Virtual memory address points-to, $\pfpointsto\vaddr\locsf\qfrac\vpts$
\end{enumerate}
\paragraph{Register points-to} The assertion $\pfpointsto\rg\rv\rpts\qfrac$ ensures the ownership of the register $\rg$ naming the register value $\rv$. The fraction $\qfrac$ with value 1 asserts the unique ownership of the register mapping, and grants update permission on it, otherwise, any value $0 < \qfrac <1$ represents partial ownership granting readonly permission on the mapping.
\paragraph{Physical-Memory points-to} The our physical memory points-to relation ($\pfpointsto\locsf\locsf\qfrac\ppts$) exhibits nested-mappings due to masking applied on the indexing address ($\locsf$). The unfolded definition of the physical mapsto 
\begin{figure}[!ht]
\[
\begin{array}{cl}
\pfpointsto\locsf\locsf\qfrac\ppts \stackrel{def}{=} & \nfpointsto{\mask\locsf\tw}{\mask\locsf\ft}\locsf\qfrac\naddr
\end{array}
\]
\caption{Physical Points-to with Nested Masking}
  \label{fig:physicalpointsto}
\end{figure}
where we simply abstract the generation of hierarchical mapping of a physical address ($\locsf$) through the different masks of the address ($\locsf|^{12}$ and $\locsf|^{52}$) as nested map resources.
\paragraph{Virtual-Memory points-to} As we see in Figure \fref{fig:}, virtual address mapping is obtained via traversal of page table, i.e. indirection provided by physical memory lookups. As expected, we define the virtual-points to relation, ($\pfpointsto\vaddr\locsf\qfrac\vpts$), in terms of multiple physical memory mappings representing the indirection shown in Figure \fref{fig:}
\begin{figure*}
\[
\begin{array}{l}
  \ppointsto\vaddr\locsf\vpts \stackrel{def}{=} \lambda \tlf\Loc.\\
  \exists_{\tlfoff\Locn \;, \tltoff\Locn \;, \tltwoff\Locn \;,\tlooff\Locn \;, \tpgoff\Loctw \;, \tlt\Loc \;, \tltw\Loc \;, \tlo\Loc \tpg\Loc} \ldotp \\
  \ulcorner \textsf{aligned } \vaddr \urcorner \star 
   \ulcorner \vaddr = \locsx :: \lfoff :: \ltoff :: 
   \ltwoff :: \looff :: \pgoff \urcorner \star\\
  \pfpointsto{\lvlsum\crt\lfoff}{\lvlbor\lt}\qfracfotsss\ppts \star 
  \pfpointsto{\lvlsum\lt\ltoff}{\lvlbor\ltw}\qfracfotss\ppts \star \\
  \pfpointsto{\lvlsum\ltw\ltwoff}{\lvlbor\lo}\qfracfots\ppts \star 
  \pfpointsto{\lvlsum\lo\looff}{\lvlbor\pg}\qfracfot\ppts \star \\
  \ppointsto{\pageptstosum\pg\pgoff}\locsf\ppts 
\end{array}
\]
\caption{Virtual Points-to Relation}
\todo[inline]{These fractions aren't quite right (though I see you added the variance between levels), I can walk you through in our meeting tomorrow.}
  \label{fig:virtualpointsto}
\end{figure*}
\section{Considering Hoare Doubles in the Context of Switching Address-Space}
An important subtlety arises with supporting \lstinline|mov|s into \lstinline|%cr3|. Consider the hypothetical rule:
\begin{mathpar}
\inferrule[Broken]{ }{
  \{P \ast cr3=r_1 \ast r=r_2 \ast [r_2](Q)\}
  \texttt{mov}~\texttt{\%cr3},~r%\lstinline|mov %cr3, r| 
  \{[r_1](P) \ast cr3=r_2 \ast r = r_2 \ast Q\}
}
\end{mathpar}
This rule captures the intuitive change of address space in a Hoare triple, rather than double, form. The problem with this is that it interacts quite poorly with the traditional frame rule and the modal flavor of virtual points-to assertions:
\begin{mathpar}
  \inferrule*[right=Frame]{
    \inferrule*[right=Broken]{ }{
    \{\mathsf{emp} \ast cr3=r_1 \ast r=r_2 \ast [r_2](Q)\}
    \texttt{mov}~\texttt{\%cr3},~r%\lstinline|mov %cr3, r| 
    \{[r_1](\mathsf{emp}) \ast cr3=r_1 \ast r=r_2 \ast Q\}
    }
  }{
    \{a\mapsto_\mathsf{v} x \ast \mathsf{emp} \ast cr3=r_1 \ast r=r_2 \ast [r_2](Q)\}
    \texttt{mov}~\texttt{\%cr3},~r%\lstinline|mov %cr3, r| 
    \{a\mapsto_\mathsf{v} x \ast [r_1](\mathsf{emp}) \ast cr3=r_1 \ast r=r_2 \ast Q\}
  }
\end{mathpar}
Notice that both the precondition and postcondition assert that $a\mapsto_\mathsf{v} x$ in the current address space, but we have no basis for concluding that address translation is preserved by the change of address space. So this derivation clearly leads to an unsound conclusion. This suggestss that the traditional frame rule and the Hoare triple presentation of the change-of-address-space rule cannot soundly coexist in the same system.
The heart of the problem is that while updating \lstinline|cr3| is \emph{physically} local, it globally changes the interpretation of virtual addresses. So it is simply unsound to frame around \lstinline|cr3| updates.

Switching to Hoare doubles resolves this problem because an under-appreciated subtlety of Hoare doubles is that typically \emph{there is no frame rule}. Instead each verification essentially includes a local frame that it passes to the next instructions (think continuation-passing style), giving each overall rule a \emph{global} (rather than local) precondition. For most rules this is not that important, but it does permit rules that have global effects on their preconditions.

This is then how we justify our actual rule for \lstinline|cr3| updates:
\begin{mathpar}
\inferrule[ChangeAddressSpace]{
  \{[r_1](P) \ast cr3=r_2 \ast r = r_2 \ast Q\}\overline{is}
}{
  \{P \ast cr3=r_1 \ast r=r_2 \ast [r_2](Q)\}
  \texttt{mov}~\texttt{\%cr3},~r;\;\overline{is}
  %\lstinline|mov %cr3, r| 
}
\end{mathpar}
Because the precondition on this rule is global, we avoid issues with framing.

If we wanted to consider a frame rule that would work for this logic, we could consider:
\begin{mathpar}
  \inferrule[Cr3Frame]{
    \{P\ast cr3=v\}\;C\;\{ Q \ast cr3=v\}
  }{
    \{R\ast P\ast cr3=v\}\;C\;\{ R \ast Q \ast cr3=v\}
  }
\end{mathpar}
By demanding that \lstinline|cr3| be held constant (or rather, at least restored to its original value) we could frame almost traditionally. In particular, this rule would work with framing around calls that might lead to address space switches, such as calling blocking operations in the kernel.

Readers familiar with dynamic frames~\cite{parkinson2011relationship} might find it useful to notice that a different perspective on this matter is that virtual points-to assertions are self-stable \emph{except} for changes in \lstinline|cr3|, so framing would then naturally require other means of holding \lstinline|cr3| constant (or saving and restoring it).
Virtual points-to assertions could be made self-stable by also giving them partial ownership over \lstinline|cr3| assertions, but this would require explicitly plumbing that ownership from \emph{all} assertions back to any place an address space change might occur; this would seem to be a far graver loss of modularity than this extra quirk in framing discussions.
