\section{Program Logic for Location Virtualization}
\label{sec:logic}
In this section, we describe our high-level reasoning principles for location virtualization whose detailed justification is shown in \sref{sec:soundness}. 
\subsection{Points-To Assertions}
\label{sec:pointsto}

We present three main points-to relations as part of utilization of separation logic \ref{} as an ambient logic:
\begin{enumerate}
\item Physical address points-to, $\pfpointsto\locsf\locsf\ppts\qfrac$
\item Register points-to, $\pfpointsto\rg\rv\rpts\qfrac$
\item Virtual memory address points-to, $\pfpointsto\vaddr\locsf\vpts\qfrac$
\end{enumerate}
\paragraph{Register points-to} The assertion $\pfpointsto\rg\rv\rpts\qfrac$ ensures the ownership of the register $\rg$ naming the register value $\rv$. The fraction $\qfrac$ with value 1 asserts the unique ownership of the register mapping, and grants update permission on it, otherwise, any value $0 < \qfrac <1$ represents partial ownership granting readonly permission on the mapping.
\paragraph{Physical-Memory points-to} The our physical memory points-to relation ($\pfpointsto\locsf\locsf\ppts\qfrac$) exhibits nested-mappings due to masking applied on the indexing address ($\locsf$). The unfolded definition of the physical mapsto 
\begin{figure}[!ht]
\[
\begin{array}{cl}
\pfpointsto\locsf\locsf\ppts\qfrac \stackrel{def}{=} & \nfpointsto{\mask\locsf\tw}{\mask\locsf\ft}\locsf\qfrac\naddr
\end{array}
\]
\caption{Physical Points-to with Nested Masking}
  \label{fig:physicalpointsto}
\end{figure}
where we simply abstract the generation of hierarchical mapping of a physical address ($\locsf$) through the different masks of the address ($\locsf|^{12}$ and $\locsf|^{52}$) as nested map resources.
\paragraph{Virtual-Memory points-to} As we see in Figure \fref{fig:}, virtual address mapping is obtained via traversal of page table, i.e. indirection provided by physical memory lookups. As expected, we define the virtual-points to relation, ($\pfpointsto\vaddr\locsf\vpts\qfrac$), in terms of multiple physical memory mappings representing the indirection shown in Figure \fref{fig:}
\begin{figure*}
\[
\begin{array}{cl}
  \pfpointsto\locsf\locsf\ppts\qfrac \stackrel{def}{=} \lambda \tlf\Loc.
  & \exists_{ \tlfoff\Locn \;, \tltoff\Locn \;, \tltwoff\Locn \;, \tlooff\Locn \;, \tpgoff\Loctw \;, \tlt\Loc \;, \tltw\Loc \;, \tlo\Loc \tpg\Loc   }
\end{array}
\]
\caption{Virtual Points-to Relation}
  \label{fig:virtualpointsto}
\end{figure*}

\begin{comment}
   Definition virt_mapsto (a:word 64) (v:word 64) (q:Qp) : word 64 -> iProp Σ :=
λ (cr:word 64) ,
     ( ⌜ aligned a ⌝ ∗
         ∃ (l3 l2 l1 page w :word 64) (pageoff:word 12) (l4off l3off l2off l1off:word 9) ,
           ⌜ (WordImpl.concat (wzero 16) (WordImpl.concat l4off (WordImpl.concat l3off (WordImpl.concat l2off (WordImpl.concat l1off pageoff))))) = a ⌝ ∗
                          (natToWord 64 ((wordToNat l4)+(8*(wordToNat l4off)))) ↦p{q} (l3  ^|  (natToWord 64 3)) ∗
                          (natToWord 64 ((wordToNat l3)+(8*(wordToNat l3off)))) ↦p{q} (l2  ^|  (natToWord 64 3)) ∗
                          (natToWord 64 ((wordToNat l2)+(8*(wordToNat l2off)))) ↦p{q} ( l1  ^|  (natToWord 64 3)) ∗
                          (natToWord 64 ((wordToNat l2)+(8*(wordToNat l1off)))) ↦p{q} ( page  ^|  (natToWord 64 3)) ∗
                          (natToWord 64 ((wordToNat page) +(wordToNat pageoff))) ↦p v )%I.
\end{comment}

\subsection{Modal Abstraction for Virtualizing Memory Locations}
\label{sec:modallocationvirtualization}
\begin{figure}
  \[
  \begin{array}{r@{\;}c@{\;}lll}
\modaldef\ell\Phi
& \logequiv &
\modaldefunfold\ell\Phi
&& \TirNameStyle{ModalLocVirt}
\\

\Phi \vdash \Theta & \rightarrow & \modaldef\ell\Phi \vdash \modaldef\ell\Theta&& \TirNameStyle{ModalLocVirtMono}
  \end{array}
  \]

\caption{Selected Logical Implications, Equivalences, and Updates}
\Description{Selected Logical Implications, Equivalences, and Updates}
\label{fig:laws}
\end{figure}


\subsection{Reasoning Rules for Selected Instructions}
\label{sec:reasoning}
\begin{figure*}
\small
\begin{mathpar}
\inferrule[Skip]{}{
  \triple\post\iskip\post
}

\inferrule[Seq]{
  \triple\pre{\instr_1}\midpoint \\
  \triple\midpoint{\instr_2}\post
}{
  \triple\pre{(\iseq{\instr_1}{\instr_2})}\post
}

\inferrule[MovRR]{}{
  % Cheat on spacing so that Move and Alloca fit on a single line.
  %(▷ (ir rsrc) ↦ᵣ{q} rv) ∗
  %(▷ (ir rdst) ↦ᵣ rvsome)
  \renewcommand{\bightriplehskip}{}
  \bightriple
    {
      \pfpointsto\rgsrc\rvsrc\rpts\qfrac \star
      \pfpointsto\rgdst\rvdst\rpts\qfrac
      \ppointsto\vaddr\locsf\vpts
    }
    {\ising{\imov\mvrr\rgsrc\rgdst}}
    {
      \pfpointsto\rgsrc\rvsrc\rpts\qfrac \star
      \pfpointsto\rgdst\rvdst\rpts\qfrac
      \ppointsto\vaddr\locsf\vpts
    }
}

\inferrule[MovMRBase]{}{
  % Cheat on spacing so that Move and Alloca fit on a single line.
 %   (▷ (ir r) ↦ᵣ{q} (num rv)) ∗
  %    (▷ (ir imreg) ↦ᵣ{q} (num maddr)) ∗
   %   (▷ (virt_mapsto maddr  mv  1 ) cr3val)
  \renewcommand{\bightriplehskip}{}
  \bightriple
    {
      \pfpointsto\rgsrc\rvsrc\rpts\qfrac \star
      \pfpointsto\rgdst\rvdst\rpts\qfrac
      \ppointsto\vaddr\locsf\vpts
    }
    {\ising{\imov\mvmro\amode\rgsrc}}
    {
      \pfpointsto\rgsrc\rvsrc\rpts\qfrac \star
      \pfpointsto\rgdst\rvdst\rpts\qfrac
      \ppointsto\vaddr\locsf\vpts
    }
}

\inferrule[MovMROff]{}{
  % Cheat on spacing so that Move and Alloca fit on a single line.
  \renewcommand{\bightriplehskip}{}
  \bightriple
    {
      \pfpointsto\rgsrc\rvsrc\rpts\qfrac \star
      \pfpointsto\rgdst\rvdst\rpts\qfrac
      \ppointsto\vaddr\locsf\vpts
    }
    {\ising{\imov\mvmro\amode\rgsrc}}
    {
      \pfpointsto\rgsrc\rvsrc\rpts\qfrac \star
      \pfpointsto\rgdst\rvdst\rpts\qfrac
      \ppointsto\vaddr\locsf\vpts
    }
}

\inferrule[MovRMBase]{}{
  % Cheat on spacing so that Move and Alloca fit on a single line.
  \renewcommand{\bightriplehskip}{}
  \bightriple
    {
      \pfpointsto\rgsrc\rvsrc\rpts\qfrac \star
      \pfpointsto\rgdst\rvdst\rpts\qfrac
      \ppointsto\vaddr\locsf\vpts
    }
    {\ising{\imov\mvrmb\rgsrc\amode}}
    {
      \pfpointsto\rgsrc\rvsrc\rpts\qfrac \star
      \pfpointsto\rgdst\rvdst\rpts\qfrac
      \ppointsto\vaddr\locsf\vpts
    }
}

\inferrule[MovRMOff]{}{
  % Cheat on spacing so that Move and Alloca fit on a single line.
  \renewcommand{\bightriplehskip}{}
  \bightriple
    {
      \pfpointsto\rgsrc\rvsrc\rpts\qfrac \star
      \pfpointsto\rgdst\rvdst\rpts\qfrac
      \ppointsto\vaddr\locsf\vpts
    }
    {\ising{\imov\mvrmo\rgsrc\amode}}
    {
      \pfpointsto\rgsrc\rvsrc\rpts\qfrac \star
      \pfpointsto\rgdst\rvdst\rpts\qfrac
      \ppointsto\vaddr\locsf\vpts
    }
}
\end{mathpar}
\caption{Reasoning Rules}
\Description{Reasoning Rules}
\label{fig:reasoning}
\end{figure*}


