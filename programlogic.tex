\section{Program Logic for Location Virtualization}
\label{sec:logic}
In this section, we describe our high-level reasoning principles for location virtualization whose detailed justification is shown in \sref{sec:soundness}. 
\subsection{Points-To Assertions}
\label{sec:pointsto}

We present three main points-to relations as part of utilization of separation logic \ref{} as an ambient logic:
\begin{enumerate}
\item Physical address points-to, $\pfpointsto\locsf\locsf\ppts\qfrac$
\item Register points-to, $\pfpointsto\rg\rv\rpts\qfrac$
\item Virtual memory address points-to, $\pfpointsto\vaddr\locsf\vpts\qfrac$
\end{enumerate}
\paragraph{Register points-to} The assertion $\pfpointsto\rg\rv\rpts\qfrac$ ensures the ownership of the register $\rg$ naming the register value $\rv$. The fraction $\qfrac$ with value 1 asserts the unique ownership of the register mapping, and grants update permission on it, otherwise, any value $0 < \qfrac <1$ represents partial ownership granting readonly permission on the mapping.
\paragraph{Physical-Memory points-to} The our physical memory points-to relation ($\pfpointsto\locsf\locsf\ppts\qfrac$) exhibits nested-mappings due to masking applied on the indexing address ($\locsf$). The unfolded definition of the physical mapsto 
\[
\begin{array}{cl}
\pfpointsto\locsf\locsf\ppts\qfrac \stackrel{def}{=} & \nfpointsto{\mask\locsf\tw}{\mask\locsf\ft}\locsf\qfrac\naddr
\end{array}
\]
where we simply abstract the generation of hierarchical mapping of a physical address ($\locsf$) through the different masks of the address ($\locsf|^{12}$ and $\locsf|^{52}$) as nested map resources.
\paragraph{Virtual-Memory points-to}
\subsection{Modal Abstraction for Virtual Memory}
\label{sec:credits}
\begin{figure}
\[\begin{array}{r@{\;}c@{\;}lll}
\modaldef
& \logequiv &
\modaldefunfold
&& \TirNameStyle{ModalLocVirt}
\end{array}\]
\caption{Selected Logical Implications, Equivalences, and Updates}
\Description{Selected Logical Implications, Equivalences, and Updates}
\label{fig:laws}
\end{figure}


\subsection{Memory Deallocation as a Ghost Operation}
\label{sec:free}

\subsection{Reasoning Rules for Selected Instructions}
\label{sec:reasoning}
\begin{figure*}
\small
\begin{mathpar}
\inferrule[Skip]{}{
  \triple\post\iskip\post
}

\inferrule[Seq]{
  \triple\pre{\instr_1}\midpoint \\
  \triple\midpoint{\instr_2}\post
}{
  \triple\pre{(\iseq{\instr_1}{\instr_2})}\post
}

\inferrule[MovRR]{}{
  \renewcommand{\bightriplehskip}{}
  \bigvtriple
    {
      \pfpointsto\rgsrc\rvsrc\qfrac\rpts \star
      \ppointsto\rgdst\rvdst\rpts
    }
    {\ising{\imov\mvrr\rgsrc\rgdst}}
    {
      \pfpointsto\rgsrc\rvsrc\qfrac\rpts \star
      \ppointsto\rgdst\rvsrc\rpts
    }
}

\inferrule[MovMRBase]{}{
  % Cheat on spacing so that Move and Alloca fit on a single line.
 %   (▷ (ir r) ↦ᵣ{q} (num rv)) ∗
  %    (▷ (ir imreg) ↦ᵣ{q} (num maddr)) ∗
   %   (▷ (virt_mapsto maddr  mv  1 ) cr3val)
  \renewcommand{\bightriplehskip}{}
  \bigvtriple
    {
      \pfpointsto\imreg\maddr\qfrac\rpts \star
      \ppointsto\rgdst\rvdst\rpts \star
      \pfpointsto\maddr\locsf\qfrac\vpts \star
    }
    {\ising{\imov\mvmro\amode\rgdst}}
    {
      \pfpointsto\imreg\maddr\qfrac\rpts \star
      \ppointsto\rgdst\locsf\rpts \star
      \pfpointsto\maddr\locsf\qfrac\vpts \star
    }
}

\inferrule[MovMROff]{}{
  % Cheat on spacing so that Move and Alloca fit on a single line.
  \renewcommand{\bightriplehskip}{}
  \bigvtriple
    {
      \ulcorner \amodeo\imreg\offs \urcorner \star 
      \pfpointsto\imreg\maddr\qfrac\rpts \star
      \ppointsto\rgdst\rvdst\rpts \star
      \pfpointsto{\plusaddr\maddr\offs}\locsf\qfrac\vpts \star
    }
    {\ising{\imov\mvmro\amode\rgdst}}
    {
      \pfpointsto\imreg\maddr\qfrac\rpts \star
      \ppointsto\rgdst\locsf\rpts \star
      \pfpointsto{\plusaddr\maddr\offs}\locsf\qfrac\vpts \star
    }
}

\inferrule[MovRMBase]{}{
  % Cheat on spacing so that Move and Alloca fit on a single line.
  \renewcommand{\bightriplehskip}{}
  \bigvtriple
    {
      \pfpointsto\imreg\maddr\qfrac\rpts \star
      \pfpointsto\rgsrc\rvsrc\qfrac\rpts \star
      \ppointsto\maddr\locsf\vpts \star
    }
    {\ising{\imov\mvrmb\rgsrc\amode}}
    {
      \pfpointsto\imreg\maddr\qfrac\rpts \star
      \pfpointsto\rgsrc\rvsrc\qfrac\rpts \star
      \ppointsto\maddr\rvsrc\vpts \star
    }
}

\inferrule[MovRMOff]{}{
  % Cheat on spacing so that Move and Alloca fit on a single line.
  \renewcommand{\bightriplehskip}{}
  \bigvtriple
      {
      \ulcorner \amodeo\imreg\offs \urcorner \star
      \pfpointsto\imreg\maddr\qfrac\rpts \star
      \pfpointsto\rgsrc\rvsrc\qfrac\rpts \star
      \ppointsto{\plusaddr\maddr\offs}\locsf\vpts \star
    }
    {\ising{\imov\mvrmo\rgsrc\amode}}
    { 
      \pfpointsto\imreg\maddr\qfrac\rpts \star
      \pfpointsto\rgsrc\rvsrc\qfrac\rpts \star
      \ppointsto{\plusaddr\maddr\offs}\rvsrc\vpts \star
    }
}
\end{mathpar}
\caption{Hoare Triples for Selected Instructions}
\Description{Reasoning Rules}
\label{fig:reasoning}
\end{figure*}


