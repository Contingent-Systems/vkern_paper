
\definecolor{mGreen}{rgb}{0,0.6,0}
\definecolor{mGray}{rgb}{0.5,0.5,0.5}
\definecolor{mPurple}{rgb}{0.58,0,0.82}
\definecolor{backgroundColour}{rgb}{0.95,0.95,0.92}

\lstdefinestyle{CStyle}{
    backgroundcolor=\color{backgroundColour},   
    commentstyle=\color{mGreen},
    keywordstyle=\color{magenta},
    numberstyle=\tiny\color{mGray},
    stringstyle=\color{mPurple},
    basicstyle=\footnotesize,
    breakatwhitespace=false,         
    breaklines=true,                 
    captionpos=b,                    
    keepspaces=true,                 
    numbers=left,                    
    numbersep=5pt,                  
    showspaces=false,                
    showstringspaces=false,
    showtabs=false,                  
    tabsize=2,
    language=C
}

\section{C Implementation of \textsf{PTE} Library}
\label{sec:cimpl}

\subsection{Page-Table-Walk}
\label{sec:ctablewalk}

\begin{lstlisting}[caption={Page-Table-Entry Library for Page-Table Traversal},label={cpagetablewalk},style=CStyle]
void pte_initialize(pte_t *entry) {
   
    // Allocate a full page for 512 8-byte entries
    pte_t *local = kalloc();
   
    // Clear the entire table. Ensures present (and reserved, etc.) bits are 0
    //memset(local, 0, PAGE_SIZE);
    entry->pfn = PTE_ADDR_TO_PFN((uintptr_t) local);

    return;
}

/*
 * pte_get_next_table - given a page table entry, returns a pointer to the base
 * of the next level of page tables, allocating a new one if necessary and if
 * told to do so
 * @param entry - the entry from which to get the next level of tables
 * @param alloc - if non-zero, if the next table is not present, allocates it
 * @return - pointer to the next page table level or, if not present and alloc
 * is not set or memory is not sufficient, NULL
 */
 pte_t *pte_get_next_table_succ(pte_t *entry) {
    pte_t *next;
   
    
    // If not already present, try to allocate
    if (!entry->present) {
      // If it shouldn't be or cannot be allocated, indicate failure
      pte_initialize(entry);
     

      // Update the entry to indicate that a new table has now been allocated.
      // Make it writable and user-accesible - if necessary, the lowest level
      // of page table can fix these access rights. Indicate the *physical*
      // address of the newly allocated structure in pfn.
      entry->writable = 1;
      entry->user_acc = 1;
      // entry->pfn = PTE_ADDR_TO_PFN((uintptr_t) next);
      entry->present = 1;
    }
    uintptr_t next_phys_addr = PTE_PFN_TO_ADDR(entry->pfn);
    uintptr_t next_virt_addr = (uintptr_t) P2V(next_phys_addr);
    next = (pte_t *) next_virt_addr;

    return next;
}


pte_t *walkpgdir_succ(pte_t *pml4, const void *va) {
  pte_t *pdp;
  pte_t *pd;
  pte_t *pt;
  uintptr_t next_phys_addr;
  uintptr_t next_virt_addr;  
  pte_t *local;
   
  pte_t *pml4_entry = &pml4[PML4EX(va)];
    
    // If not already present, try to allocate
    if (!pml4_entry->present) {
      // If it shouldn't be or cannot be allocated, indicate failure
      // Allocate a full page for 512 8-byte entries
      local = kalloc();
   
      // Clear the entire table. Ensures present (and reserved, etc.) bits are 0
      //memset(local, 0, PAGE_SIZE);
      
      *(&pdp) = local;

      // Update the entry to indicate that a new table has now been allocated.
      // Make it writable and user-accesible - if necessary, the lowest level
      // of page table can fix these access rights. Indicate the *physical*
      // address of the newly allocated structure in pfn.
      pml4_entry->writable = 1;
      pml4_entry->user_acc = 1;
      pml4_entry->pfn = PTE_ADDR_TO_PFN((uintptr_t) pdp);
      pml4_entry->present = 1;
    }
    next_phys_addr = PTE_PFN_TO_ADDR(pml4_entry->pfn);
    next_virt_addr = (uintptr_t) P2V(next_phys_addr);
    pdp = (pte_t *) next_virt_addr;

    //= pte_get_next_table_succ(pml4_entry);

    pte_t *pdp_entry = &pdp[PDPEX(va)];

     // If not already present, try to allocate
    if (!pdp_entry->present) {
      // If it shouldn't be or cannot be allocated, indicate failure
      // Allocate a full page for 512 8-byte entries
      local = kalloc();
   
      // Clear the entire table. Ensures present (and reserved, etc.) bits are 0
      //memset(local, 0, PAGE_SIZE);
      
      *(&pd) = local;

      // Update the entry to indicate that a new table has now been allocated.
      // Make it writable and user-accesible - if necessary, the lowest level
      // of page table can fix these access rights. Indicate the *physical*
      // address of the newly allocated structure in pfn.
      pdp_entry->writable = 1;
      pdp_entry->user_acc = 1;
      pdp_entry->pfn = PTE_ADDR_TO_PFN((uintptr_t) pd);
      pdp_entry->present = 1;
    }
    next_phys_addr = PTE_PFN_TO_ADDR(pdp_entry->pfn);
    next_virt_addr = (uintptr_t) P2V(next_phys_addr);
    pd = (pte_t *) next_virt_addr;
    
   

    pte_t *pd_entry = &pd[PDEX(va)];
  // If not already present, try to allocate
    if (!pd_entry->present) {
      // If it shouldn't be or cannot be allocated, indicate failure
      // Allocate a full page for 512 8-byte entries
      local = kalloc();
   
      // Clear the entire table. Ensures present (and reserved, etc.) bits are 0
      //memset(local, 0, PAGE_SIZE);
      
      *(&pt) = local;

      // Update the entry to indicate that a new table has now been allocated.
      // Make it writable and user-accesible - if necessary, the lowest level
      // of page table can fix these access rights. Indicate the *physical*
      // address of the newly allocated structure in pfn.
      pd_entry->writable = 1;
      pd_entry->user_acc = 1;
      pd_entry->pfn = PTE_ADDR_TO_PFN((uintptr_t) pt);
      pd_entry->present = 1;
    }
    next_phys_addr = PTE_PFN_TO_ADDR(pd_entry->pfn);
    next_virt_addr = (uintptr_t) P2V(next_phys_addr);
    pt = (pte_t *) next_virt_addr;
    
    pt = pte_get_next_table_succ(pd_entry);

    return &pt[PTEX(va)];
}
\end{lstlisting}

\subsection{Mapping a Page}
\label{sec:cmapapage}

\begin{lstlisting}[caption={Mapping a Single Page},label={cmapapage},style=CStyle]
complstatus_t vaspace_mappage(
    pte_t *pml4,
    void *va,
    uintptr_t pa,
    uint64_t perm
) {
    char *a;
    pte_t *pte;

    a = (char *)PGROUNDUP((uintptr_t)va);
    
    pte = walkpgdir_succ(pml4, a);
    if (!pte->present){
      pte->writable     = PTE_WRITEABLE_SET(perm);
      pte->user_acc     = PTE_USER_ACC_SET(perm);
      pte->writethrough = PTE_WRITETHRU_SET(perm);
      pte->nocache      = PTE_NOCACHE_SET(perm);
      pte->pfn          = PTE_ADDR_TO_PFN(pa);
      pte->nx           = PTE_NX_SET(perm);
      pte->present      = 1;
    }

    return COMPL_SUCCESS;
}
\end{lstlisting}


\subsection{Address-Space-Switch}
\label{sec:caspaceswitch}

\begin{lstlisting}[caption={Loading \textsf{cr3} for Address-Space Switch},label={caspace},style=CStyle]
void pte_load(pte_t *pml4) {
    lcr3(V2P(pml4));
}
\end{lstlisting}
