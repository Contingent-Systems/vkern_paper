\section{Background}
\label{sec:background}
\subsection{Virtual Memory Managers}
\label{sec:backgroundonvmm}
\subsection{Modal Logic}
\label{sec:backgroundonmodallogic}
Work on modal logic extends back for decades with many widely-varied variations; we discuss here only the pieces most relevant to our work.
Broadly speaking modal logics incorporate \emph{modal operators}, which take as arguments a proposition expected to be true in another time~\cite{pnueli1977temporal}, place~\cite{gordon2019modal,goranko1996hierarchies,areces2001hybrid,gargov1993modal}, or circumstance~\cite{hintikka1962knowledge,halpern1985guide}, and result in a proposition true in the \emph{current} time, place, or circumstance at which the truth of the use of the modal operator is being evaluated. Classic examples include modal necessity $\square P$ describing that $P$ is \emph{necessarily} true, $\mathsf{G}~P$ meaning $P$ is true \emph{globally} (i.e., forever from this time onwards), or $K_i(P)$ describing that a particular participant $i$ \emph{knows} that $P$ is true. The latter is an example of \emph{multimodal} logic, where there is an indexed family of modalities (modal operators) parameterized by some dimension of interest (there, participants).
A closely related variant of multimodal logic is \emph{dynamic logic}~\cite{pratt1976semantical}, a logic of weakest preconditions~\cite{dijkstra-75} which works with modalities of the form $[p](P)$, which states that \emph{in the current program state}, \emph{if} program $p$ is run then afterwards $P$ will hold (modulo non-termination).
This same idea is used to encode Hoare triples in the Iris program logic~\cite{krebbers2017essence}, using the same encoding as in Pratt's original presentation, where a Hoare triple $\{P\}C\{Q\}$ is encoded as $P\rightarrow[C](Q)$. Unlike classic work on dynamic logic~\cite{harel2000dynamic}, Iris applies these ideas in a \emph{substructural} setting (separation logic) where distributivity laws over substructural connectives must be considered. For both the dynamic modality and the later modality $\blacktriangleright$ used for guarded recursive predicates, these modalities satisfy axioms of the form $M(P)\ast Q\vdash M(P\ast Q)$, but not the reverse.
Iris is not the first combination of modal and substructural logic~\cite{dovsen1992modal,restall1993modalities,d1997grafting,kamide2002kripke,licata2017fibrational}, but is certainly the most heavily tested.
A hallmark of a unary logical operator $M$ being a modality is if $M$ satisfies a property akin to $(\upvarphi\rightarrow\psi)\rightarrow M(\upvarphi)\rightarrow M(\psi)$, which roughly states that modus ponens holds under the modality.\footnote{Afficionados of modal logic will note that this property is not quite Axiom K (which requires the initial implication to also hold under $M$), but follows from K and a necessitation rule $\upvarphi\rightarrow M(\upvarphi)$. Non-necessitive modalities typically satisfy the weaker property we call out above.}
% Most pertinent to our work is that Iris itself has a modalities built into the logic to solve problems with step-indexing / guardedness (the later modality) and general infrastructure for duplicable assumptions / resources (the persistent modality, reminiscent of linear logic's exponential). And of course the central concept in the use of Iris as a program logic is the use of the weakest precondition modality, which like our modalities is defined \emph{within} the logic itself, rather than being part of the base logic.

Another pillar of our work is hybrid logic, a branch of modal logic with \emph{names} (called \emph{nominals}) for states in Kripke models~\cite{blackburn1995hybrid,goranko1996hierarchies,areces2001hybrid,gargov1993modal}, in contrast to the typical flavor of modality which refers to an \emph{unspecified} (in the assertion) set of alternative circumstances. 

Our work draws on ideas from hybrid modal logics, applied in the context of Iris's higher-order separation logic~\cite{jung2018iris}. We define our modalities and prove their rules directly within Iris, so there is no model-theoretic novelty in our work. Instead, we show that targeted use of modalities combining the ideas of \emph{named resources} with the power of substructural reasoning enable clear specification of general reasoning principles for data structures. Along the way we take advantage of the fact that modalities need not be injections between the \emph{same} logics, but in fact the propositions that modal operators contextualize can come from \emph{other} logics; in our case we exploit the fact that Iris itself is parameterized by a choice of step-indexed~\cite{ahmed-appel-virga-02} \textsf{BI}-algebra~\cite{ohearn1999bunched} to allow convenient specifications in embedded data-structure-relative logics.
