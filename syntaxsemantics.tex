\definecolor{dkgreen}{rgb}{0,0.6,0}
\definecolor{ltblue}{rgb}{0,0.4,0.4}
\definecolor{dkviolet}{rgb}{0.3,0,0.5}

% lstlisting coq style (inspired from a file of Assia Mahboubi)
\lstdefinelanguage{Coq}{ 
    % Anything betweeen $ becomes LaTeX math mode
    mathescape=true,
    % Comments may or not include Latex commands
    texcl=false, 
    % Vernacular commands
    morekeywords=[1]{Section, Module, End, Require, Import, Export,
        Variable, Variables, Parameter, Parameters, Axiom, Hypothesis,
        Hypotheses, Notation, Local, Tactic, Reserved, Scope, Open, Close,
        Bind, Delimit, Definition, Let, Ltac, Fixpoint, CoFixpoint, Add,
        Morphism, Relation, Implicit, Arguments, Unset, Contextual,
        Strict, Prenex, Implicits, Inductive, CoInductive, Record,
        Structure, Canonical, Coercion, Context, Class, Global, Instance,
        Program, Infix, Theorem, Lemma, Corollary, Proposition, Fact,
        Remark, Example, Proof, Goal, Save, Qed, Defined, Hint, Resolve,
        Rewrite, View, Search, Show, Print, Printing, All, Eval, Check,
        Projections, inside, outside, Def},
    % Gallina
    morekeywords=[2]{forall, exists, exists2, fun, fix, cofix, struct,
        match, with, end, as, in, return, let, if, is, then, else, for, of,
        nosimpl, when},
    % Sorts
    morekeywords=[3]{Type, Prop, Set, true, false, option},
    % Various tactics, some are std Coq subsumed by ssr, for the manual purpose
    morekeywords=[4]{pose, set, move, case, elim, apply, clear, hnf,
        intro, intros, generalize, rename, pattern, after, destruct,
        induction, using, refine, inversion, injection, rewrite, congr,
        unlock, compute, ring, field, fourier, replace, fold, unfold,
        change, cutrewrite, simpl, have, suff, wlog, suffices, without,
        loss, nat_norm, assert, cut, trivial, revert, bool_congr, nat_congr,
        symmetry, transitivity, auto, split, left, right, autorewrite},
    % Terminators
    morekeywords=[5]{by, done, exact, reflexivity, tauto, romega, omega,
        assumption, solve, contradiction, discriminate},
    % Control
    morekeywords=[6]{do, last, first, try, idtac, repeat},
    % Comments delimiters, we do turn this off for the manual
    morecomment=[s]{(*}{*)},
    % Spaces are not displayed as a special character
    showstringspaces=false,
    % String delimiters
    morestring=[b]",
    morestring=[d],
    % Size of tabulations
    tabsize=3,
    % Enables ASCII chars 128 to 255
    extendedchars=false,
    % Case sensitivity
    sensitive=true,
    % Automatic breaking of long lines
    breaklines=false,
    % Default style fors listings
    basicstyle=\small,
    % Position of captions is bottom
    captionpos=b,
    % flexible columns
    columns=[l]flexible,
    % Style for (listings') identifiers
    identifierstyle={\ttfamily\color{black}},
    % Style for declaration keywords
    keywordstyle=[1]{\ttfamily\color{dkviolet}},
    % Style for gallina keywords
    keywordstyle=[2]{\ttfamily\color{dkgreen}},
    % Style for sorts keywords
    keywordstyle=[3]{\ttfamily\color{ltblue}},
    % Style for tactics keywords
    keywordstyle=[4]{\ttfamily\color{dkblue}},
    % Style for terminators keywords
    keywordstyle=[5]{\ttfamily\color{dkred}},
    %Style for iterators
    %keywordstyle=[6]{\ttfamily\color{dkpink}},
    % Style for strings
    stringstyle=\ttfamily,
    % Style for comments
    commentstyle={\ttfamily\color{dkgreen}},
    %moredelim=**[is][\ttfamily\color{red}]{/&}{&/},
    literate=
    {\\forall}{{\color{dkgreen}{$\forall\;$}}}1
    {\\exists}{{$\exists\;$}}1
    {<-}{{$\leftarrow\;$}}1
    {=>}{{$\Rightarrow\;$}}1
    {==}{{\code{==}\;}}1
    {==>}{{\code{==>}\;}}1
    %    {:>}{{\code{:>}\;}}1
    {->}{{$\rightarrow\;$}}1
    {<->}{{$\leftrightarrow\;$}}1
    {<==}{{$\leq\;$}}1
    {\#}{{$^\star$}}1 
    {\\o}{{$\circ\;$}}1 
    {\@}{{$\cdot$}}1 
    {\/\\}{{$\wedge\;$}}1
    {\\\/}{{$\vee\;$}}1
    {++}{{\code{++}}}1
    {~}{{\ }}1
    {\@\@}{{$@$}}1
    {\\mapsto}{{$\mapsto\;$}}1
    {\\hline}{{\rule{\linewidth}{0.5pt}}}1
    %
}[keywords,comments,strings]

\section{Machine State \& Syntax}
\label{sec:syntax}
To develop our core logical ideas, we instantiate \textsf{Iris} with a simple language for streams of instructions, 
and a logical machine model corresponding to execution of x86-64 assembly instructions with virtual memory enabled on the 
CPU.

\subsection{Registers and Memory}
Programs we demonstrate in this paper requires accessing two types of computer resource: registers and memory.
 A register identifier, $\reg$, is chosen from a fixed finite set of register identifiers, $\regset$. 
We use these identifiers to access the register values, $\regval \in \regvaltype$.
\begin{figure}[t]
\newcommand{\commentary}[1]{ & \text{\small\it #1} \\}
\[
  \begin{array}{r@{\;}c@{\;}l}
    \loc & \in & \Loc \\
    \reg & \in & \regset \\
    \regval & \in & \regvaltype \\
    \val & ::= & \vunit

\\
    \instrs & ::= &
    \begin{array}[t]{@{}l@{\hspace{10mm}}l@{}}
    \begin{array}[t]{@{}ll@{}}
      \iskip
                   \commentary{no-op}
      \iseq\instr\instrs
                   \commentary{sequencing}
      % \ising\instr
                   % \commentary{executing}             
    \end{array}
    %&
    %\begin{array}[t]{@{}ll@{}}
     % \ialloc\lval\allocsize
      %             \commentary{heap allocation}
    %\end{array}
    \end{array}
    \\

    \ectx & ::= &
      \hole \mid
      \iseq\ectx\instrs 
    \\
  \end{array}
\]
\caption{Syntax}
\Description{Syntax}
\label{fig:syntax}
\end{figure}
Unlike registers, we do not abstract the memory indices as a special type but instead, for the sake of clarity and ease of representation, we show differently masked machine words, $\loc \in \Loc$, with the subscripts showing the length of a 64-bit machine word after masking, e.g. $\kw{w}_{12}$ is a 12-bit resource which can be obtained after masking 52-bit of a 64-bit word.
This simple syntax in Figure \ref{fig:syntax} includes a simple syntax for values because our language is purely \textit{imperative}, i.e. an instruction does not return a value. Intentionally, we restrict any stream of instruction, $\instrs$, to eventually reduce to $\iskip$ instruction. Our syntax capture the no-op ``$\iskip$'' and the sequencing construct ``$\iseq{\instr}{\instrs}$'' in their standard forms.
\subsection{State}
\label{sec:state}
%\todo[inline]{I don't think we need or want to describe the operational semantics in detail in the submission. It adds a lot of formalization that doen'st necessarily contribute to reader understanding. We should describe the runtime \emph{state} $\sigma$ of memory and registers, since that is directly used in interpreting assertions. But the reduction rules will mostly add mental overhead here.}
%\todo[inline,color=red]{Colin says: On the contrary, we must describe assembly instructions in enough detail for readers
%who aren't already assembly programmers to decode our examples! We don't have to talk about details of the formalization
%of the semantics, but we do need to talk about basic instructions we cover, basic patterns (like destination on the left),
%and just a bit about how all memory accesses implement Figure \ref{fig:pagetables}'s semantics.}
We represent the machine state mainly as a finite map of registers to register values and a map of masked physical memory addresses to 64-bit physical memory values. Together with default \textsf{CPU} instantiation, we end up having our state with following pieces:
\begin{itemize}
\item A constant cpu initialized with default value: $\sigma.\mathcal{C}$
\item A register map: $\sigma.\mathcal{R}: \kw{greg} \rightarrow_{\textrm{fin}} \kw{regval} $
\item A memory map: $\sigma.\mathcal{M}: \Locft \rightharpoonup_{\textrm{fin}} (\Loctw \rightharpoonup_{\textrm{fin}} \Locsf )$
\end{itemize}
As one might have already anticipated from the syntax we introduce, we do not bind any value of an evaluated expression. All the indices and accessed values are treated as globally referenced. In align with this design choice, our expression is a stream of instructions, which is not evaluated to a value to be bound, but changes the machine state through the indices (e.g. $\kw{r}\in\kw{greg}$) -- to the global maps.
\subsection{Instructions}
\label{sec:instructions}
Page-table walk shown in Figure \ref{fig:pagetables} is realized with physical memory \texttt{load}s and \textsf{store}s for the entries in the page-tables which are simply modelled as lookup and updates to an instance of a memory map with 64 bit entries. In the address translation,
\begin{lstlisting}[language=Coq]
 Definition translate (rtv: word 64) (w: word 64): word 64 + MemFail :=
      l4e $\leftarrow$ translate_top_level $\sigma.\mathcal{M}$ rtv w; 
      l3e $\leftarrow$ translate_from_l4e $\sigma.\mathcal{M}$ l4e w; 
      l2e $\leftarrow$ translate_from_l3e $\sigma.\mathcal{M}$ l3e w;
      l1e $\leftarrow$ translate_from_l2e $\sigma.\mathcal{M}$ l2e w;
      addr $\leftarrow$ translate_from_l1e $\sigma.\mathcal{M}$ l1e w;
      concat (shift_dropn addr 3 three_lt64) low3.
\end{lstlisting}
we see a physical memory $\mathsf{load}$ (i.e. physical memory map ($\sigma.\mathcal{M}$) for each level) for each level of page-table-walk concluded with alignment of the returned address for each \textit{successful} page-table-walk traversal which is what our reasoning principles only consider.

As expected \textsf{store} would be the update on the memory map entry
\todo[inline,color=yellow]{Colin, could you mention that we simplified the maps here to get rid of MemFail?}
After defining the translation in terms of physical load and store action, now we can define our memory \lstinline|mov| instructions
\[
\begin{array}{l}
  \hbox{(\TirNameStyle{WriteToMemFromReg})} \qquad
  \textsf{mov mem64 srcreg64} \\
  \hbox{(\TirNameStyle{WriteToRegFromMem})} \qquad 
  \textsf{mov reg64 mem64}
  \end{array}
  \]
  which are realized by first translating the virtual-memory address to store a value from a register (\TirNameStyle{WriteToMemFromReg}) or load a value at the translated address to be stored in a register (\TirNameStyle{WriteToRegFromMem}). More \lstinline|mov| instructions operating with memory addresses can be found in the submission artifact for different addressing schemes and data sizes.

  Register instructions, on the other hand, are implemented as $\sigma.\mathcal{R}$ lookup and store
\[
\begin{array}{l}
  \hbox{(\TirNameStyle{WriteToRegFromReg})} \qquad 
   \textsf{mov ireg64 ireg64} \\
  \hbox{(\TirNameStyle{WriteToCtlRegFromReg})} \qquad 
  \textsf{mov creg64 ireg64}\\
    \hbox{(\TirNameStyle{WriteToRegFromCtlReg})} \qquad 
  \textsf{mov ireg64 creg64}
  \end{array}
\]
for different types of registers (e.g. control and general).
%
%\subsection{Machine State under Address Translation}
%\label{sec:selectedinstrsemantics}
%Althought we give the complete set of operational semantics rule in \sref{appendix:movops}, It is worth building the intuition on the the way some of these rules bahave in the context of address translation. In fact $\readlval\maddr{\storememstar\crval}\locsf$ is unfoled into multiple physical memory lookups for the final page address retrieval -- i.e. address translation traversal as shown in Figure \ref{fig:pagetables}:
%\begin{itemize}
%\item top-level-address translation: with the given root address (64-bit $\crval$) of the address space, performs the address translation, handling the first level (to get the PML4 entry) itself. The next level table address is computed with the fetched PML4 offset value which exhibits itself as 9-bit offset in $\kw{maddr}$. \mytodo{iso: put move}
%\item translating from PML4 entry: performs the second level of address translation, to retrieve starting at the PML4 table entry, and interprets the PML4 entry that references a Page-Directory-Pointer Table (PDPT). We obtain the PDPT offset which exhibits itself as 9 bit offset in $\kw{maddr}$ to obtain the address of the next level page directory table (PDT) \mytodo{iso: put move}
%\item translating from PD entry: performs the third level of address translation, to retrieve starting at the PDP table entry, and interprets a PDPTE that references a PD table. Likewise, we obtain the PD offset which exhibits itself as 9 bit offset in $\kw{maddr}$ to obtain the address of the next level page directory table (PT) \mytodo{iso:put move }
%\item translate from PT entry: performs final level of address translation, starting from the PT entry, with a given 12 bit page offset, we can compute the physical address referencing $\locsf$ \mytodo{ismo: put mov}
%\end{itemize}
