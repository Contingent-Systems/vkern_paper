\section{Syntax}
\label{sec:syntax}
\subsection{Registers, Memory Indices}
There exist two pieces of resources we access in our model: registers and memory. A register identifier, $\reg$, is chosen from a fixed finite set of register identifiers,$\regset$. We use these identifiers to access the register values, $\regval \in \regvaltype$. Unlike registers, we do not abstract the memory indices as a special type but instead, for the sake of clarity and ease of representation, we show differently masked machine words, $\loc \in \Loc$, with the subscripts showing the length of a 64-bit machine word after masking, e.g. $\kw{w}_{12}$ is a 12-bit resource which can be obtained after masking 52-bit of a 64-bit word.
This simple syntax in \ref{fig:syntax} includes a simple syntax for values because our language is purely \textit{imperative}, i.e. an instruction does not return a value. Intentionally, we restrict any stream of instruction, $\instrs$, to eventually reduce to $\iskip$ instruction which is injected into unit.
\begin{figure}[!ht]
\newcommand{\commentary}[1]{ & \text{\small\it #1} \\}
\[
  \begin{array}{r@{\;}c@{\;}l}
    \loc & \in & \Loc \\
    \reg & \in & \regset \\
    \regval & \in & \regvaltype \\
    \val & ::= & \vunit

\\
    \instrs & ::= &
    \begin{array}[t]{@{}l@{\hspace{10mm}}l@{}}
    \begin{array}[t]{@{}ll@{}}
      \iskip
                   \commentary{no-op}
      \iseq\instr\instrs
                   \commentary{sequencing}
      \ising\instr
                   \commentary{executing}             
    \end{array}
    %&
    %\begin{array}[t]{@{}ll@{}}
     % \ialloc\lval\allocsize
      %             \commentary{heap allocation}
    %\end{array}
    \end{array}
    \\

    \ectx & ::= &
      \hole \mid
      \iseq\ectx\instrs 
    \\
  \end{array}
\]
\caption{Syntax}
\Description{Syntax}
\label{fig:syntax}
\end{figure}

\subsection{State}
\label{state}
